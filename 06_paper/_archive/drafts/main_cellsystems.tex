\documentclass[review]{elsarticle}

\usepackage{amsmath,amssymb}
\usepackage{graphicx}
\usepackage{booktabs}
\usepackage{multirow}
\usepackage{hyperref}
\usepackage{subcaption}
\usepackage{lineno}
\usepackage{xcolor}

\modulolinenumbers[5]

\journal{Cell Systems}

\begin{document}

\begin{frontmatter}

\title{CellSwarm: LLM-Driven Cell Agent Simulation for Tumor Immune Microenvironment Modeling and Immunotherapy Prediction}

\author[1]{Dawn Meng\corref{cor1}}
\ead{email@example.com}
\cortext[cor1]{Corresponding author}
\address[1]{Department of Computer Science, University, City, Country}

\begin{abstract}
Agent-based models (ABMs) are widely used to simulate multicellular systems, yet their reliance on hand-crafted if-else rules limits their ability to capture the complex signal pathway crosstalk governing cell behavior in the tumor immune microenvironment (TIME). We present CellSwarm, a simulation framework in which each cell is an independent AI agent whose behavioral decisions are made by a large language model (LLM) acting as a signal pathway integrator. Each cell agent maintains persistent state including 20+ signaling pathway activation scores, cell cycle phase, metabolic status, and a memory stream of past experiences. We initialize 1,000 cell agents from real single-cell RNA-seq data (GSE176078, triple-negative breast cancer) and systematically compare LLM-driven decisions against rule-based and random baselines within the same framework. LLM-driven cells produce population dynamics closer to observed clinical proportions, maintain higher immune cell viability (CD8+ T cells: 260 vs.\ 249 for rules, 12 for random), and exhibit emergent behaviors including hypoxia-driven migration and coordinated immune responses. The framework achieves high reproducibility (tumor count CV = 0.4\% across 5 seeds) while running 1,000 cells in 3.5 minutes. We demonstrate translational utility through immunotherapy simulations: PD-1 blockade produces a clear dose-response relationship (97\% tumor reduction at full dose, IC$_{50} \approx$ 40\%), early treatment outperforms late intervention, and combination therapies (PD-1 + CTLA-4, PD-1 + TGF-$\beta$) show synergistic effects reducing tumors to 13--15 cells from 735. Validation against GSE169246 immunotherapy clinical data confirms directional consistency of treatment responses. CellSwarm represents the first application of LLM agents to multicellular biological simulation, establishing a new paradigm for computational biology that eliminates the rule engineering bottleneck of traditional ABMs.
\end{abstract}

\begin{keyword}
large language model \sep agent-based modeling \sep tumor immune microenvironment \sep cell simulation \sep immunotherapy \sep signal pathway integration
\end{keyword}

\end{frontmatter}

\linenumbers

% ============================================================
\section{Introduction}
% ============================================================

The tumor immune microenvironment (TIME) is a complex ecosystem where tumor cells, immune cells, and stromal cells interact through hundreds of molecular signals \citep{binnewies2018understanding}. Understanding these interactions is critical for developing effective immunotherapies, yet the sheer complexity of signal pathway crosstalk makes computational modeling extremely challenging.

Agent-based models (ABMs) have emerged as a powerful paradigm for simulating multicellular systems \citep{an2017optimization}. Tools such as PhysiCell \citep{ghaffarizadeh2018physicell}, CompuCell3D \citep{swat2012multi}, Chaste \citep{mirams2013chaste}, and BioDynaMo \citep{breitwieser2021biodynamo} enable researchers to model individual cell behaviors and observe emergent tissue-level phenomena. However, these frameworks share a fundamental limitation: cell behavioral rules must be manually specified as deterministic if-else logic. When a cell simultaneously receives conflicting signals---for example, a CD8+ T cell encountering both activating IFN-$\gamma$ and suppressive TGF-$\beta$ in the presence of PD-L1---the modeler must hard-code the resolution priority. This approach requires extensive domain expertise, fails to generalize across scenarios, and cannot capture the nuanced, context-dependent integration that real cells perform through their signaling networks.

Recent advances in large language models (LLMs) have demonstrated remarkable capabilities in complex reasoning and decision-making. The Generative Agents framework \citep{park2023generative} showed that LLM-powered agents can exhibit emergent social behaviors in simulated environments. Multi-agent systems such as AgentVerse \citep{chen2023agentverse} and MetaGPT \citep{hong2023metagpt} have further demonstrated collective intelligence from LLM agent collaboration. However, no prior work has applied LLM agents to biological cell simulation---a search of arXiv for ``LLM cell simulation'' returns zero results as of early 2026.

We propose CellSwarm, a framework that bridges this gap by treating each cell as an LLM-driven agent. Our key insight is that the LLM serves not as an autonomous decision-maker, but as a \textbf{signal pathway integrator}---a computational analog of the cell's internal signaling network that resolves multi-pathway crosstalk to determine the appropriate biological response. This paradigm eliminates the rule engineering bottleneck while preserving biological plausibility.

Our contributions are:
\begin{enumerate}
    \item We introduce the first LLM-driven cell agent simulation framework, where LLMs replace hard-coded rules for signal pathway integration in multicellular simulation.
    \item We demonstrate that LLM-driven cells initialized from real scRNA-seq data reproduce key tumor immune microenvironment phenotypes and outperform rule-based alternatives in immune cell preservation.
    \item Through systematic comparison with published ABM tools and internal ablation studies, we show that CellSwarm achieves competitive biological realism with zero rule engineering.
    \item We simulate immunotherapy interventions including dose-response, treatment timing, and combination therapy, with results validated against clinical data (GSE169246).
\end{enumerate}

% ============================================================
\section{Results}
% ============================================================

\subsection{CellSwarm Framework Overview}

CellSwarm operates as a single-process simulation with three core modules (Figure~1): (1) a \textbf{Cell Agent Layer} where each of 1,000 cells maintains persistent state including position, energy, cell cycle phase, 20+ signaling pathway activation scores, and a memory stream; (2) an \textbf{Environment Engine} implementing reaction-diffusion dynamics for oxygen, glucose, and signaling molecules (IFN-$\gamma$, IL-2, TGF-$\beta$, PD-L1) on a 100$\times$100 grid; and (3) an \textbf{LLM Integration Layer} that encodes cell state into structured prompts and receives JSON-formatted behavioral decisions.

Each simulation step follows: environment update $\rightarrow$ treatment application $\rightarrow$ cell sensing $\rightarrow$ pathway computation $\rightarrow$ LLM/rule decision $\rightarrow$ action execution $\rightarrow$ interaction resolution. The LLM is invoked only when a cell's pathway complexity score exceeds a threshold, with rule-based fallback for simple cases, achieving $\sim$7,000 LLM calls per 30-step simulation.

\subsection{Baseline Validation with Real Data}

We initialized 1,000 cell agents from the GSE176078 dataset \citep{wu2021single}, a comprehensive single-cell RNA-seq atlas of human breast cancer, with type proportions matching the original dataset (75\% tumor, 10\% CD8+ T, 5\% Treg, 10\% macrophage). Over 30 simulation steps, the LLM-driven simulation reproduced several key biological phenomena (Figure~2):

\begin{itemize}
    \item Tumor growth following a Gompertz curve with carrying capacity
    \item Cold tumor phenotype with high tumor-to-immune ratio
    \item Immune evasion in high-PD-L1 regions
    \item Hypoxia-driven cell migration toward oxygenated areas
\end{itemize}

\subsection{Systematic Method Comparison}

We compared three decision modes within the same framework---LLM, rule-based, and random---each repeated 5 times with different seeds (Figure~3, Table~\ref{tab:comparison}).

\begin{table}[h]
\centering
\caption{Cell population comparison at step 30 (mean $\pm$ SD, n=5)}
\label{tab:comparison}
\begin{tabular}{lcccccc}
\toprule
Method & Tumor & CD8+ T & Treg & Mac & Total & CV \\
\midrule
LLM & 732$\pm$3 & 260$\pm$0 & 89$\pm$0 & 199$\pm$0 & 1280$\pm$3 & 0.4\% \\
Rules & 749$\pm$0 & 249$\pm$0 & 89$\pm$0 & 199$\pm$0 & 1286$\pm$0 & 0.0\% \\
Random & 809$\pm$9 & 12$\pm$3 & 86$\pm$1 & 196$\pm$2 & 1102$\pm$15 & 1.1\% \\
\bottomrule
\end{tabular}
\end{table}

Key findings: (1) LLM preserves 260 CD8+ T cells versus 249 for rules and only 12 for random, demonstrating superior immune regulation; (2) LLM produces fewer tumors (732 vs.\ 749) suggesting more effective immune-mediated control; (3) random decisions are catastrophic for immune cells (95\% T cell loss); (4) LLM achieves high reproducibility (CV = 0.4\%) comparable to deterministic rules.

\subsection{LLM Decision Mechanism Analysis}

Analysis of LLM decision patterns reveals several insights (Figure~4): (1) decision distributions are cell-type-specific and evolve over time, with CD8+ T cells shifting from patrol to attack as tumor burden increases; (2) pathway activation profiles show clear type-specific signatures (e.g., high TCR/PD-1 in CD8+ T, high PI3K/AKT in tumor); (3) LLM decisions exhibit higher Shannon entropy (3.2 bits) than rules (1.5 bits), indicating greater behavioral diversity; (4) the LLM selectively resolves complex multi-pathway conflicts that rule-based systems handle with fixed priorities.

\subsection{Immunotherapy Simulation}

We simulated PD-1 blockade immunotherapy with five dose levels, four treatment windows, and two combination regimens (Figure~5, Table~\ref{tab:treatment}).

\begin{table}[h]
\centering
\caption{Treatment simulation results (step 30)}
\label{tab:treatment}
\begin{tabular}{lccc}
\toprule
Treatment & Tumor & Change & CD8+ T \\
\midrule
No treatment & 735 & --- & 260 \\
PD-1 25\% & 572 & $-$22\% & 260 \\
PD-1 50\% & 201 & $-$73\% & 260 \\
PD-1 75\% & 77 & $-$90\% & 260 \\
PD-1 100\% & 25 & $-$97\% & 260 \\
\midrule
PD-1 + CTLA-4 & 15 & $-$98\% & 260 \\
PD-1 + TGF-$\beta$ & 13 & $-$98\% & 261 \\
\bottomrule
\end{tabular}
\end{table}

The dose-response curve exhibits a sigmoidal shape with IC$_{50} \approx$ 40\%, consistent with pharmacological dose-response relationships. Treatment timing matters: starting at step 5 yields 37 remaining tumors versus 186 at step 20. Combination therapies show synergistic effects: PD-1 + TGF-$\beta$ reduces tumors to 13 (98\% reduction), outperforming PD-1 monotherapy at 75\% strength (77 tumors).

\subsection{Comparison with Published Methods}

We systematically compared CellSwarm with five published tools across multiple dimensions (Figure~6): PhysiCell \citep{ghaffarizadeh2018physicell}, CompuCell3D \citep{swat2012multi}, Chaste \citep{mirams2013chaste}, BioDynaMo \citep{breitwieser2021biodynamo}, and Generative Agents \citep{park2023generative}. CellSwarm uniquely combines: (1) zero manual rule engineering (vs.\ 15--25 rules for traditional ABMs); (2) 20 signaling pathways (vs.\ 2--5); (3) emergent behavior capability (shared only with Generative Agents); (4) treatment simulation with dose-response prediction; and (5) agent memory for context-dependent decisions. The trade-off is computational speed (210s vs.\ 2--30s for rule-based tools) and scale (1K vs.\ up to 1M cells).

\subsection{Clinical Validation with GSE169246}

We validated CellSwarm predictions against the GSE169246 dataset \citep{zhang2021single}, comprising 489,490 cells from 22 TNBC patients with pre- and post-immunotherapy scRNA-seq data (Figure~7). Key validation results:

\begin{itemize}
    \item Treatment response direction: both real data and simulation show Treg proportion decrease post-treatment (real: 5.3\%$\rightarrow$4.6\%; simulated: 6.9\%$\rightarrow$decreased)
    \item Dose-response relationship: simulation IC$_{50} \approx$ 40\% is consistent with clinical response rates (KEYNOTE-522: $\sim$64\% pCR)
    \item Combination superiority: simulation confirms combo $>$ monotherapy, consistent with clinical trial results (CheckMate-227, KEYNOTE-598)
    \item Early treatment advantage: simulation shows earlier intervention yields better outcomes, consistent with neoadjuvant therapy evidence
\end{itemize}

% ============================================================
\section{Discussion}
% ============================================================

\subsection{LLM as Signal Pathway Integrator}

The central innovation of CellSwarm is reconceptualizing the LLM's role: rather than an autonomous decision-maker, it serves as a computational analog of the cell's internal signaling network. When a CD8+ T cell simultaneously receives IFN-$\gamma$ (activating), TGF-$\beta$ (suppressive), and PD-L1 (inhibitory) signals, the LLM integrates these conflicting inputs using its trained knowledge of immunology to produce a contextually appropriate response. This is fundamentally different from rule-based systems where the modeler must pre-specify the resolution of every possible signal combination.

\subsection{Emergent Behaviors}

Several biologically meaningful behaviors emerge without explicit programming: (1) hypoxia-driven migration---cells in low-O$_2$ regions spontaneously migrate toward oxygenated areas; (2) immune exclusion zones---tumor cells create PD-L1-rich regions that T cells avoid; (3) coordinated immune response---CD8+ T cells concentrate near tumor boundaries. These emergent properties arise from the LLM's ability to make contextually appropriate decisions based on its training knowledge, rather than from hard-coded behavioral rules.

\subsection{Limitations and Future Directions}

Several limitations should be noted: (1) current simulations use 1,000 cells; real tumors contain millions, requiring hierarchical approaches or fine-tuned smaller models for scaling; (2) the 2D grid does not capture full spatial complexity; (3) temporal resolution requires calibration with longitudinal data; (4) LLM biological knowledge may contain inaccuracies, addressable through RAG with curated pathway databases; (5) while GLM-4-Flash is currently free, scaling requires cost optimization.

Future directions include: RAG integration with KEGG/Reactome databases, fine-tuning specialized models, 3D extension, multi-model ensemble approaches, and application to drug screening and personalized treatment prediction.

% ============================================================
\section{Methods}
% ============================================================

\subsection{Cell Agent Architecture}

Each cell agent maintains: (1) identity (unique ID, cell type from \{Tumor, CD8+ T, Treg, Macrophage\}); (2) state vector $\mathbf{s}_i = [\mathbf{p}_i, e_i, \phi_i, \mathbf{a}_i, \mathbf{w}_i]$ where $\mathbf{p}_i \in \mathbb{R}^2$ is position, $e_i \in [0,1]$ is energy, $\phi_i \in \{G0, G1, S, G2, M\}$ is cell cycle phase, $\mathbf{a}_i$ is activation/exhaustion, and $\mathbf{w}_i \in \mathbb{R}^{20}$ is pathway activation; (3) memory stream recording recent decisions and outcomes, following the architecture of \citet{park2023generative}.

\subsection{Signal Pathway Computation}

Pathway activation scores are computed from local environment concentrations using sigmoid transfer functions. For CD8+ T cells: $\text{TCR} = \sigma(\text{tumor\_neighbors} \cdot 0.5 + \text{activation} \cdot 0.3)$, $\text{PD-1} = \sigma(\text{PD-L1} \cdot 1.2 + \text{exhaustion} \cdot 0.5)$, etc. The pathway complexity score $C = \min(1, n_{\text{active}}/N + \text{Var}(\mathbf{w}))$ determines whether LLM or rule-based decision is used.

\subsection{Environment Engine}

A 100$\times$100 grid maintains concentration fields evolving via:
\begin{equation}
    \frac{\partial c}{\partial t} = D \nabla^2 c + S(\mathbf{x}) - \lambda c
\end{equation}
where $D$ is diffusion coefficient, $S(\mathbf{x})$ is cell-mediated production, and $\lambda$ is decay rate.

\subsection{LLM Integration}

Cell state is encoded into structured prompts including cell type, energy, cycle phase, pathway scores, local environment, and neighbor information. The LLM (GLM-4-Flash, temperature 0.3) returns JSON decisions specifying action, parameters, and biological reasoning. Concurrent API calls ($\sim$50 parallel) enable efficient batch processing.

\subsection{Treatment Simulation}

Five treatment modalities are implemented: anti-PD-1 (reduces PD-L1 field and tumor immune evasion, activates exhausted T cells), anti-CTLA-4 (enhances T cell co-stimulation, suppresses Treg), anti-TGF-$\beta$ (reduces immunosuppressive microenvironment, promotes M2$\rightarrow$M1 polarization), and two combinations. Treatment is applied at configurable start step with adjustable strength $s \in [0,1]$.

\subsection{Data Sources}

GSE176078 \citep{wu2021single}: single-cell RNA-seq atlas of human breast cancer for initialization. GSE169246 \citep{zhang2021single}: TNBC immunotherapy scRNA-seq (489,490 cells, 22 patients, pre/post treatment) for validation. Cell type annotation uses marker gene scoring (EPCAM/KRT19 for tumor, CD8A/CD8B for CD8+ T, FOXP3/IL2RA for Treg, CD68/CD163 for macrophage).

\subsection{Statistical Analysis}

All experiments repeated 5 times with seeds 42, 123, 456, 789, 2024. Reproducibility assessed by coefficient of variation (CV). Shannon entropy quantifies decision diversity. Results reported as mean $\pm$ SD.

% ============================================================
\section*{Data and Code Availability}
% ============================================================

CellSwarm source code and all experimental data are available at [repository URL]. Raw scRNA-seq data: GSE176078 and GSE169246 from NCBI GEO.

\section*{Acknowledgments}

We thank the authors of GSE176078 and GSE169246 for making their data publicly available.

\bibliographystyle{elsarticle-num}
\bibliography{references}

\end{document}
