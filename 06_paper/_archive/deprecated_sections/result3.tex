Results 1 and 2 established that CellSwarm achieves comparable fidelity to hand-coded rules in baseline simulations and can generalize across cancer types and treatment conditions. However, in those experiments, Agent and Rules produced qualitatively similar outcomes---raising the question of whether LLM-driven decision-making offers any mechanistic advantage over explicit rule programming. We hypothesized that the key differentiator would emerge under genetic perturbations that require multi-step causal reasoning: scenarios where the perturbed gene does not directly participate in the cell's decision-making pathways but instead exerts its effect through intermediate signaling cascades.

Critically, the rule-based engine in CellSwarm evaluates CD8+ T cell attack decisions using exactly four checkpoint pathway variables: net activation = TCR + CD28 $-$ PD-1 $-$ CTLA-4 $>$ 0.5. Genes that do not feed into these four variables are, by design, invisible to the Rules engine. This architectural constraint provides a controlled experimental setup: if we perturb a gene outside the rule boundary, any response from the Agent mode must arise from the LLM's ability to reason beyond the explicit decision logic.

We therefore simulated five gene knockouts (KOs) spanning two categories. "Direct" KOs targeted immune checkpoint receptors whose pathway values are explicitly used in the attack threshold: PD-1 KO and CTLA-4 KO. "Indirect" KOs targeted genes whose effects on tumor immunity are mediated through intermediate signaling outside the rule boundary: IFN-$\gamma$ KO (disrupts JAK-STAT1 signaling, reducing CD8+ T cell cytotoxicity), TGF-$\beta$ KO (removes SMAD-mediated immunosuppression), and TP53 KO (alters tumor cell apoptosis and proliferation). Each KO was simulated under both Agent and Rules modes with three independent seeds (Fig. 4).

For direct KOs, both Agent and Rules responded with substantial tumor reduction (Fig. 4A). PD-1 KO reduced the tumor ratio to 0.67 $\pm$ 0.08 (Agent) and 0.47 $\pm$ 0.05 (Rules), representing 27.9\% and 45.0\% decreases relative to their respective baselines (P = 0.032 and P = 0.001). CTLA-4 KO produced similar effects: 0.63 $\pm$ 0.07 (Agent, $-$32.1\%) and 0.44 $\pm$ 0.07 (Rules, $-$48.4\%). These results confirm that both decision-making modes correctly interpret the removal of inhibitory checkpoint signals as a license for enhanced immune attack. Notably, Rules produced more aggressive tumor clearance than Agent in both direct KOs, consistent with the deterministic, threshold-based nature of rule-based decisions: once the inhibitory signal is removed, every CD8+ T cell that meets the activation threshold attacks without hesitation.

The critical divergence emerged with indirect KOs (Fig. 4B). Under IFN-$\gamma$ KO, Agent simulations showed tumor expansion (ratio 1.09 $\pm$ 0.04, a 17.5\% increase over baseline), while Rules simulations were completely unchanged from baseline (ratio 0.85 $\pm$ 0.02, Δ = 0.0\%). The difference between Agent and Rules under IFN-$\gamma$ KO was highly significant (P = 0.002). TGF-$\beta$ KO and TP53 KO showed the same pattern: Agent produced modest tumor ratio increases of 6.8\% and 14.8\% respectively, while Rules remained at exactly the baseline value in all three seeds (Fig. 4B). A heatmap summarizing the percentage change in tumor ratio across all five KOs and both modes made this dichotomy visually striking (Fig. 4C): the Rules column showed exactly 0.0\% change for all three indirect KOs, while the Agent column showed graded positive responses.

The IFN-$\gamma$ KO result is the most mechanistically informative (Fig. 4D). IFN-$\gamma$ is not directly referenced in the rule-based attack threshold; its effect on CD8+ T cell function is mediated through the JAK-STAT1 signaling cascade, which amplifies T cell cytotoxicity and promotes antigen presentation. When IFN-$\gamma$ signaling is ablated, the rule-based engine---which evaluates only the four checkpoint pathways (TCR, CD28, PD-1, CTLA-4)---sees no change in its input variables and therefore produces identical behavior. The LLM-driven agent, by contrast, has access to the full pathway state including the IFN-$\gamma$/JAK-STAT1 axis and can reason about its downstream consequences. Tumor count trajectories under IFN-$\gamma$ KO showed Agent simulations trending upward from approximately 200 to 230 tumor cells over 30 steps, while Rules trajectories were indistinguishable from untreated baseline (Fig. 4D).

We examined whether the perturbations altered the immune microenvironment composition by computing the CD8+ T cell to Treg ratio under each KO condition in Agent simulations (Fig. 4E). The CD8/Treg ratio remained relatively stable across all conditions (range: 4.00--4.11), suggesting that the perturbation effects on tumor growth were mediated primarily through changes in per-cell killing efficiency rather than shifts in immune cell population dynamics. This is consistent with the short simulation duration (30 steps), which may be insufficient for population-level remodeling to manifest.

Finally, we plotted the Agent versus Rules sensitivity for each KO as a scatter plot (Fig. 4F), where the x-axis represents the Rules percentage change and the y-axis represents the Agent percentage change. Direct KOs (PD-1, CTLA-4) clustered in the lower-left quadrant, indicating that both modes responded with tumor reduction, with Rules showing larger effects. Indirect KOs (IFN-$\gamma$, TGF-$\beta$, TP53) clustered along the y-axis at x $\approx$ 0, indicating that only Agent responded while Rules remained inert. This separation provides a clear visual summary of the fundamental mechanistic difference: rule-based models respond only to perturbations that directly alter their input variables, whereas LLM-driven agents can propagate perturbation effects through implicit causal reasoning over the full signaling network.

These findings establish genetic perturbation sensing as a capability unique to LLM-driven cell agents. While rule-based models excel at executing well-defined decision logic---and indeed produce stronger responses to direct pathway perturbations---they are fundamentally limited to the variables explicitly encoded in their rules. LLM-driven agents, by leveraging the biological knowledge embedded in their training data and the structured pathway information provided by the knowledge bases, can infer the downstream consequences of perturbations that lie outside the rule-based decision boundary. This capability is particularly relevant for applications in drug target discovery and genetic screen interpretation, where the perturbation of interest often acts through indirect, multi-step mechanisms.
