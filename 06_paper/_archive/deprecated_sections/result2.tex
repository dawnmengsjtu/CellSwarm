Having established that CellSwarm recapitulates TNBC microenvironment composition (Fig. 2), we next asked whether the framework could generalize to other cancer types without retraining or reprogramming the LLM backbone. We constructed knowledge base entries for five additional cancer types---CRC-MSI-H, CRC-MSS, melanoma, NSCLC, and high-grade serous ovarian cancer---based on published literature characterizing their TME phenotypes (Bagaev et al., Cancer Cell, 2021). Importantly, unlike the TNBC simulations that were calibrated against single-cell RNA sequencing data (GSE176078), the knowledge bases for these five cancer types were derived from literature-reported immune infiltration patterns and cell-type proportions rather than from patient-level transcriptomic data. We therefore evaluated cross-cancer simulations on the basis of qualitative directional consistency with established immunological phenotypes, rather than quantitative concordance with a specific reference dataset.

We simulated each cancer type using the same CellSwarm framework, LLM backbone (DeepSeek), and simulation parameters, changing only the cancer-specific knowledge base entry. Each cancer was run with three independent seeds. The six cancer types exhibited markedly different tumor dynamics (Fig. 3B): CRC-MSI-H showed the most aggressive tumor clearance (tumor ratio 0.07 $\pm$ 0.02), while TNBC showed the least tumor regression (0.93 $\pm$ 0.09). Melanoma (0.27 $\pm$ 0.03), NSCLC (0.23 $\pm$ 0.01), CRC-MSS (0.26 $\pm$ 0.01), and ovarian cancer (0.18 $\pm$ 0.08) fell between these extremes.

To assess whether the simulations captured the well-established distinction between immunologically "hot" and "cold" tumors, we computed the CD8+ T cell to Treg ratio across all six cancer types (Fig. 3C). Ranking cancers from highest to lowest CD8/Treg ratio yielded CRC-MSI-H (4.65 $\pm$ 0.05), TNBC (4.03 $\pm$ 0.05), melanoma (3.48 $\pm$ 0.05), NSCLC (1.43 $\pm$ 0.00), ovarian (1.10 $\pm$ 0.01), and CRC-MSS (1.00 $\pm$ 0.00). This ordering is consistent with the known immunological classification: microsatellite-unstable colorectal cancers, triple-negative breast cancers, and melanomas are generally considered immune-hot, while microsatellite-stable colorectal cancers, ovarian cancers, and a subset of NSCLCs are immune-cold or immune-excluded. Across three immune features (CD8/Treg ratio, CD8+ T cell proportion, and NK cell proportion), hot tumors scored significantly higher than cold tumors (P $<$ 0.001 for all three features, two-sided t-test).

The CRC-MSI-H versus CRC-MSS comparison provided a particularly informative internal control, as these two subtypes arise from the same tissue but differ fundamentally in their immune microenvironment (Fig. 3D). CRC-MSI-H simulations produced a CD8/Treg ratio of 4.65 compared to 1.00 for CRC-MSS, a CD8+ T cell proportion of 40.0\% versus 12.0\%, and an NK cell proportion of 7.9\% versus 4.6\%. Conversely, CRC-MSS showed higher macrophage infiltration (26.1\% vs. 17.8\%), consistent with the M2-polarized, immunosuppressive macrophage phenotype characteristic of microsatellite-stable tumors. A summary heatmap of four immune features across all six cancer types confirmed that the hot-to-cold gradient was consistently captured across multiple dimensions (Fig. 3E).

We next evaluated whether CellSwarm could predict differential responses to immune checkpoint inhibitors. We simulated treatment with anti-PD-1 and anti-CTLA-4 antibodies, each administered at two time points: early (step 5, representing early intervention) and late (step 15, representing delayed treatment). Treatment was modeled by modifying the PD-1 or CTLA-4 pathway activation values in the cell agent's signaling network at the time of administration. Both Agent and Rules simulations showed tumor reduction relative to untreated baseline upon anti-PD-1 treatment (Fig. 3F): Agent achieved a tumor ratio of 0.78 $\pm$ 0.04 (early) and 0.83 $\pm$ 0.05 (late), compared to baseline 0.93 $\pm$ 0.09. Rules-based simulations showed similar reductions (0.73 $\pm$ 0.03 early, 0.76 $\pm$ 0.02 late). Anti-CTLA-4 treatment produced more modest effects in Agent simulations (0.91 $\pm$ 0.09 early, 0.90 $\pm$ 0.03 late), while Rules showed stronger responses (0.76 $\pm$ 0.05 early, 0.78 $\pm$ 0.02 late).

Examination of tumor count trajectories under anti-PD-1 early treatment revealed a characteristic biphasic response (Fig. 3G): tumor cells initially expanded during the first 10--15 steps, then underwent progressive regression in the later phase, with one Agent simulation declining from a peak of 310 to 153 cells by step 30. This pattern is reminiscent of the delayed response kinetics observed clinically with immune checkpoint inhibitors, where tumor regression often follows an initial period of apparent progression.

Early treatment consistently produced lower tumor ratios than late treatment for anti-PD-1 (0.78 vs. 0.83), though the difference did not reach statistical significance (P = 0.30; Fig. 3H). For anti-CTLA-4, early and late treatments produced nearly identical outcomes (0.91 vs. 0.90, P = 0.91), suggesting that the timing effect may be drug-specific and dependent on the mechanism of action.

Finally, we compared simulated tumor reduction rates with published clinical response rates (Fig. 3I). Across all seeds and timing conditions, Agent simulations produced a mean tumor reduction of 19.4\% for anti-PD-1, compared to a clinical objective response rate of approximately 21\% for pembrolizumab in TNBC (KEYNOTE-012). For anti-CTLA-4, the simulated reduction was 9.2\%, compared to a clinical response rate of approximately 12\% for ipilimumab in melanoma. While these comparisons are approximate---simulated tumor reduction and clinical ORR measure different endpoints---the concordance in magnitude suggests that CellSwarm captures the relative efficacy ranking of checkpoint inhibitors without any explicit training on clinical outcome data.

These results demonstrate two capabilities that are absent from rule-based simulations: the ability to generalize across cancer types by simply swapping knowledge base entries, and the capacity to produce treatment responses that qualitatively match clinical observations. However, we note that the cross-cancer results should be interpreted as directional validation rather than quantitative prediction, given the absence of patient-level ground truth data for the five non-TNBC cancer types.
