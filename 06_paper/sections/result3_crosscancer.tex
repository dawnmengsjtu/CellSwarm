CellSwarm recapitulates TNBC microenvironment composition (\figref{fig:2}). The next question was whether the framework could generalize to other cancer types without retraining or reprogramming the LLM backbone. We assembled knowledge base entries for five additional cancers (CRC-MSI-H, CRC-MSS, melanoma, NSCLC, and high-grade serous ovarian cancer) drawing on published literature that characterizes their TME phenotypes (\figref{fig:3}A). A key distinction from the TNBC simulations: these five knowledge bases were derived from literature-reported immune infiltration patterns and cell-type proportions, not from patient-level transcriptomic data (the TNBC runs were calibrated against GSE176078). We therefore evaluated cross-cancer simulations on qualitative directional consistency with established immunological phenotypes rather than quantitative concordance with a specific reference dataset.

Each cancer type was simulated using the same framework, LLM backbone (DeepSeek), and parameters; only the cancer-specific knowledge base entry changed. Three independent seeds per cancer. The resulting cell-type composition profiles separated cleanly (\figref{fig:3}B): CRC-MSI-H showed high CD8$^+$ T cell infiltration, ovarian cancer exhibited elevated Treg proportions, and macrophages dominated CRC-MSS---all consistent with known biology. Tumor count dynamics over 30 steps confirmed that immunologically hot tumors (CRC-MSI-H, melanoma) underwent greater immune-mediated control, while TNBC showed the least regression (\figref{fig:3}C).

The CD8$^+$ T cell to Treg ratio across all six cancer types provided a direct test of the hot-versus-cold distinction (\figref{fig:3}D). Ranking from highest to lowest: CRC-MSI-H (4.65 $\pm$ 0.06), TNBC (4.03 $\pm$ 0.06), melanoma (3.48 $\pm$ 0.06), NSCLC (1.43 $\pm$ 0.00), ovarian (1.10 $\pm$ 0.02), CRC-MSS (1.00 $\pm$ 0.00). The gradient matched expectations. A normalized immune landscape heatmap captured finer patterns as well: high NK activity in melanoma, elevated macrophage infiltration in NSCLC (\figref{fig:3}E; Table~S2). Rules-based simulations, by contrast, produced nearly invariant tumor ratios across all six types (TR = 0.04--0.07), failing to separate hot from cold tumors (Figure~S2). Cross-cancer generalization, it appears, requires the contextual reasoning that only the LLM provides.

We next asked whether CellSwarm could predict differential responses to immune checkpoint inhibitors. Three treatments (anti-PD-1, anti-CTLA-4, and anti-TGF$\beta$) were each administered at two time points: early (step 5) and late (step 15). Treatment was modeled by modifying the corresponding pathway activation values in the cell agent's signaling network at the time of administration. Both Agent and Rules reduced tumor burden relative to untreated baseline upon anti-PD-1 treatment (\figref{fig:3}F): Agent reached a tumor ratio of 0.78 $\pm$ 0.04 (early) and 0.83 $\pm$ 0.05 (late), compared to a baseline of 0.93 $\pm$ 0.09. Anti-CTLA-4 effects were more modest in Agent simulations (0.91 $\pm$ 0.09 early, 0.90 $\pm$ 0.03 late). Anti-TGF$\beta$ treatment overestimated clinical response by roughly 16-fold (simulated ORR 60.3\% vs.\ clinical $\sim$4\%), likely because the knowledge base entry conflates TGF-$\beta$'s pleiotropic effects into a simplified direct anti-tumor mechanism (Figure~S3). Full treatment results for all three drugs, timings, and modes appear in Table~S3.

Tumor count trajectories under treatment followed a biphasic pattern (\figref{fig:3}G): tumor cells expanded during the first 10--15 steps, then regressed progressively. This kinetic profile is reminiscent of the delayed response observed clinically with checkpoint inhibitors, where regression often follows an initial period of apparent progression.

Early treatment consistently yielded lower tumor ratios than late treatment for anti-PD-1 (0.78 vs.\ 0.83), though the difference did not reach statistical significance (P = 0.30; \figref{fig:3}H). Simulated tumor reduction rates relative to untreated baseline showed partial concordance with published clinical response rates: anti-PD-1 17.6\% vs.\ clinical 21\% (KEYNOTE-012); anti-CTLA-4 showed a more modest 3.3\% reduction, below the clinical 12\% (ipilimumab in melanoma), likely reflecting the simulation's conservative decision bias discussed below. These comparisons are approximate (simulated tumor reduction and clinical ORR measure different endpoints), but the concordance suggests that CellSwarm captures the relative efficacy ranking of checkpoint inhibitors without any explicit training on clinical outcome data.

Capabilities absent from rule-based simulations emerge here: generalization across cancer types by swapping knowledge base entries alone, and treatment responses that qualitatively match clinical observations.
