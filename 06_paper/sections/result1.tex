Simulating the tumor microenvironment (TME) demands capturing how hundreds of cells autonomously interpret local signals, update internal states, and act on context-dependent cues, all simultaneously. Agent-based models (ABMs) have long addressed this challenge by encoding cell behaviors as hand-written rules, an approach that faithfully reproduces known biology but cannot generalize beyond its programmed logic. A CD8$^+$ T cell in a rule-based ABM will attack a tumor cell if, and only if, the conditions its designer anticipated are met. We asked whether large language models (LLMs), whose training corpus spans the biomedical literature, could replace these static rules with a reasoning engine capable of interpreting biological contexts it was never explicitly programmed to handle. The result is CellSwarm, a multi-agent framework in which every cell in the TME operates as an autonomous LLM-driven agent (\figref{fig:1}).

The distinction from conventional ABMs is architectural. Where a traditional model evaluates a fixed conditional (``if checkpoint score $>$ threshold, then kill''), CellSwarm presents the LLM with the T cell's activation state, the local cytokine milieu, checkpoint pathway engagement, and a compressed history of prior interactions. The action that emerges is not retrieved from a lookup table; it is inferred from the confluence of these inputs. This means the same agent can, in principle, respond to microenvironmental configurations that were never specified during model construction.

Internally, each agent maintains three representational layers. A persistent state vector encodes the cell's condition at each time step: cell cycle phase (G0/G1/S/G2/M), energy reserves, activation status, and exhaustion markers. Alongside this, a signaling module tracks 14 canonical pathways---immune activation (TCR, CD28), checkpoint axes (PD-1, CTLA-4), metabolism (mTOR, AMPK, HIF-1$\alpha$), cytokine cascades (IFN-$\gamma$/JAK-STAT1, IL-2/JAK-STAT5, TGF-$\beta$/SMAD, NF-$\kappa$B), proliferative programs (PI3K/AKT, MAPK/ERK), and apoptosis (caspase)---each represented as a continuous activation value that updates with local signal concentrations and cell--cell contacts. A memory stream completes the architecture: a sliding window of the five most recent actions and up to twenty landmark events (first antigen encounter, exhaustion onset, and similar transitions), giving the agent temporal context that shapes its future decisions.

The simulation loop is a perception--reasoning--action cycle. At each step, an agent samples its neighborhood: six diffusible signals (O$_2$, glucose, IFN-$\gamma$, IL-2, PD-L1, TGF-$\beta$), the identities and states of adjacent cells, and its own internal vector. These observations are assembled into a structured prompt, augmented with entries retrieved from domain-specific knowledge bases curated from published literature (cancer-type parameters, pathway definitions, drug mechanisms), whose construction is detailed in STAR Methods. The LLM processes this prompt and commits to one of five atomic actions: proliferate, migrate, secrete cytokines, undergo apoptosis, or remain quiescent. That choice propagates through the local environment via signal diffusion on a 500 $\times$ 500 grid, altering the context that neighboring agents will perceive on the next step.

A direct consequence of this design is that biological knowledge enters the system through knowledge base entries rather than hard-coded parameters, so simulating a different cancer type requires only swapping those entries at initialization. No retraining, no code changes, no parameter sweeps. The following sections test whether this zero-shot generalization holds in practice across six cancer types.

To isolate what the LLM contributes, we built two non-LLM baselines that share every component of the simulation (grid geometry, signal diffusion, cell initialization) except the decision engine. Rules mode substitutes a hand-coded rule set encoding canonical immune--tumor logic: CD8$^+$ T cells attack when net checkpoint activation exceeds 0.5; tumor cells proliferate when energy permits and no immune threat is detected; other types follow analogous heuristics. Random mode draws actions uniformly at random. The comparison is strict: any behavioral difference between Agent and these controls arises solely from the LLM's reasoning over biological context.

We evaluate CellSwarm along five axes: baseline fidelity against single-cell transcriptomic data from triple-negative breast cancer (\figref{fig:2}), cross-cancer generalization and treatment response (\figref{fig:3}), genetic perturbation sensing (\figref{fig:4}), multi-model robustness and ablation (\figref{fig:5}), and mechanistic dissection of the emergent dynamics (\figref{fig:6}).
