Tumors are not isolated masses of malignant cells but dynamic ecosystems composed of tumor cells, immune cells, stromal cells, and vascular and lymphatic networks. The classical hallmarks framework emphasized the systemic, multi-stage nature of acquired tumor capabilities, and subsequent revisions incorporated immune evasion, inflammation, metabolic reprogramming, and microenvironment remodeling as additional dimensions, reinforcing the view that tumor progression and treatment response depend critically on microenvironmental coupling \citep{Hanahan2000, Hanahan2011}. Research centered on the tumor microenvironment (TME) has further established that cellular composition, spatial architecture, diffusion gradients, and signaling interaction networks collectively determine tumor evolutionary trajectories and therapeutic accessibility \citep{Quail2017}. The ``immune contexture''---encompassing immune cell types, functional states, and spatial organization---correlates tightly with clinical outcome and varies substantially across tumor types and individuals \citep{Fridman2012}. Immune checkpoint blockade (ICB), which unleashes anti-tumor immunity by relieving inhibitory pathway engagement, has produced breakthrough responses across multiple cancer types \citep{Pardoll2012}. Yet both clinical trials and real-world evidence show that the fraction of patients who benefit remains limited: the immune-inflamed, immune-excluded, and immune-desert subtypes of the tumor immune microenvironment dictate sensitivity and resistance to immunotherapy \citep{Binnewies2018}. Key translational challenges include the scarcity of transferable preclinical models, insufficient mechanistic interpretability, a combinatorial treatment space too vast to search systematically, and reproducibility problems driven by inter-patient heterogeneity \citep{Hegde2020}.

Triple-negative breast cancer (TNBC) illustrates these difficulties. Although PD-1/PD-L1 inhibitors show clinical activity in selected subgroups, overall response rates remain modest and are strongly dependent on baseline immune status and microenvironment structure \citep{Nanda2016, Adams2019}. Suppressive mechanisms in the TME are typically multifactorial; metabolic competition and nutrient/oxygen limitation are important drivers of effector T cell exhaustion and decline of key effector molecules such as IFN-$\gamma$ \citep{Chang2015}. Meanwhile, large-scale TME subtyping studies suggest that relatively conserved microenvironment subtypes exist across cancer types and can predict immunotherapy response, but how these statistical associations map onto executable cell-level behavioral rules---and how to perform mechanistic reasoning and counterfactual testing at the individual level---remains without a unifying framework \citep{Bagaev2021}.

Single-cell transcriptomics and spatial omics have sharply increased the resolution at which we can dissect the TME, enabling characterization of tumor ecosystems at the level of cell types, states, and interactions, and uncovering complex crosstalk networks between tumor cells and immune/stromal compartments \citep{Tirosh2016}. In solid tumors such as breast cancer, single-cell and spatially resolved data resources now systematically capture immune infiltration patterns, tumor cell state lineages, and their spatial organization, providing a data foundation for individualized mechanistic modeling \citep{Wu2021}. Multi-modal microenvironment studies have further shown immune cell remodeling and spatial heterogeneity at the tissue scale, emphasizing that spatial--functional coupling is necessary for reliable inference of treatment response \citep{Klemm2020}.

On the modeling side, multiscale and agent-based tumor simulations can explicitly represent individual cells, local interactions, and diffusion/mechanical constraints, supporting interpretable ``rule-to-phenotype'' reasoning and enabling exploration of treatment strategies, dosing regimens, and combination therapy spaces \citep{Wang2015}. Open-source platforms such as PhysiCell provide efficient simulation kernels for 3D multicellular systems with extensible phenotype modules, lowering the engineering barrier to building reusable models \citep{Ghaffarizadeh2018}. However, current agent-based models remain heavily dependent on researcher-specified rules, parameters, and interaction structures. When transferring across datasets or patients, rule extraction, parameter calibration, and uncertainty characterization become bottlenecks that limit application to large-scale individualized inference and automated model discovery.

In parallel, the rapid advances in large language models (LLMs) have opened new approaches to knowledge organization, rule generation, and tool invocation through natural language interfaces. Models exemplified by GPT-4 show strong general capabilities in multi-task reasoning, code generation, and complex instruction following \citep{OpenAI2023}. Retrieval-augmented generation (RAG) couples external knowledge bases with generative models, providing a path toward traceable factual grounding and dynamic knowledge updates \citep{Lewis2020}. The ReAct paradigm further interleaves reasoning traces with executable actions, supporting iterative planning and error correction through environment interaction \citep{Yao2022}. Related work on generative agents has shown that LLMs combined with memory, reflection, and planning modules can produce credible multi-agent behaviors and emergent phenomena in complex settings \citep{Park2023}.

Building on these advances, we propose \cellswarm{}, an LLM-driven multi-agent modeling framework for the tumor immune microenvironment. \cellswarm{} organizes single-cell and spatial data together with biological priors---signaling pathways, cell--cell interactions, and metabolic constraints---into retrievable knowledge bases, and during simulation the LLM performs rule selection, parameter suggestion, and experimental design within an interpretable mechanistic world. This enables inference of immunotherapy response, counterfactual testing of key mechanisms, and systematic search over candidate intervention strategies. We validate \cellswarm{} across six cancer types spanning the immunological spectrum (\figref{fig:2}--\figref{fig:3}), show that it senses indirect genetic perturbations that rule-based models cannot detect (\figref{fig:4}), establish robustness across multiple LLM backends (\figref{fig:5}), and dissect the mechanistic basis of emergent simulation dynamics (\figref{fig:6}).
