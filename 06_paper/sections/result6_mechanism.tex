The preceding sections evaluated CellSwarm's outputs (cell-type proportions, tumor dynamics, perturbation responses) against external benchmarks. We now turned inward, examining the simulation's internal dynamics to understand how LLM-driven agents produce biologically plausible behaviors and where the framework's limitations originate.

Realistic tumor simulation demands appropriate cell cycle control: most cells in solid tumors are quiescent, with only a small fraction actively proliferating. Cell cycle phase distributions across the three modes confirmed this (\figref{fig:6}A). Agent maintained 85.1\% of cells in G0 (quiescent), the remainder distributed across G1 (2.2\%), S (5.3\%), G2 (5.9\%), and M (1.5\%). Rules produced a similar profile (82.4\% G0). Random showed only 48.3\% G0 with excessive entry into proliferative phases---explaining the uncontrolled tumor expansion observed in earlier experiments. The LLM correctly infers that most cells should remain quiescent under homeostatic conditions, a behavior that emerges from contextual reasoning rather than explicit programming.

Baseline environmental signal levels across 14 pathways showed a consistent pattern (\figref{fig:6}B). Agent and Rules maintained similar signal profiles despite using fundamentally different decision-making mechanisms. Random showed aberrant signal accumulation across multiple pathways, consistent with the uncoordinated behaviors that produce unrealistic TME dynamics.

The anti-TGF$\beta$ treatment failure identified in Results 3 warranted a closer look (\figref{fig:6}C). The simulated tumor reduction (60.3\%) exceeded the clinical objective response rate ($\sim$5\%) by 16-fold. The Drug Library entry for anti-TGF$\beta$ encodes direct anti-tumor effects not observed clinically: TGF-$\beta$ blockade in patients primarily modulates the immune microenvironment rather than killing tumor cells directly, but the knowledge base entry conflates these mechanisms. This is a specific, correctable knowledge base deficiency, not a fundamental limitation of the LLM reasoning approach.

The IFN-$\gamma$ KO experiment (Results 4) provided the clearest window into how LLM-driven agents integrate pathway context. Environmental signal changes under IFNG\_KO relative to baseline (\figref{fig:6}D) showed the expected reduction in IFN-$\gamma$ signaling, but the perturbation also propagated downstream: TNF-$\alpha$ and IL-2 signaling both diminished, consistent with IFN-$\gamma$'s role in amplifying pro-inflammatory cytokine cascades through JAK-STAT1 activation. The LLM-driven agents propagate knockout effects through biologically plausible signaling cascades rather than treating each pathway as independent.

Treatment-induced signal dynamics reinforced this picture (\figref{fig:6}E). Both anti-PD-1 and anti-CTLA-4 elevated IFN-$\gamma$ signaling over the simulation course, consistent with checkpoint blockade restoring T cell effector function and subsequent IFN-$\gamma$ secretion. The temporal profile showed a gradual increase beginning 5--10 steps after treatment, matching the expected delay between checkpoint release and downstream cytokine production.

Treatment effects on cell cycle distribution (\figref{fig:6}F) showed that anti-PD-1 increased tumor cell entry into apoptotic pathways while maintaining immune cell quiescence. This selective effect is biologically appropriate: checkpoint blockade enhances immune killing of tumor cells without directly altering immune cell proliferation kinetics. The LLM-driven agents captured this distinction.

An immune remodeling heatmap across all seven gene knockouts (\figref{fig:6}G) showed that each perturbation produced a distinct cell-type-specific response signature. PD1\_KO and CTLA4\_KO enhanced CD8$^+$ T cell proportions, consistent with checkpoint release. IFNG\_KO reduced both NK and CD8$^+$ T cell proportions, reflecting IFN-$\gamma$'s broad immunostimulatory role. TP53\_KO and BRCA1\_KO primarily affected tumor cell dynamics with minimal immune remodeling, consistent with their cell-intrinsic oncogenic functions. These distinct signatures confirm that LLM-driven agents generate perturbation-specific responses grounded in the known biology of each targeted gene.

Environmental signal profiles between drug treatments and their corresponding gene knockouts provided mechanistic support for the phenocopy observations in \figref{fig:4}F--H (\figref{fig:6}H). PD1\_KO and anti-PD-1 produced highly similar signal profiles, confirming that the genetic and pharmacological interventions converge on the same downstream signaling state, a signal-level explanation for the high phenocopy correlation ($r = 0.86$) observed at the cell-type composition level.

Knowledge base query patterns across experimental conditions (\figref{fig:6}I) confirmed active utilization by Agent mode. Each baseline run queried approximately 26 drug entries, 15 pathway entries, and 64 perturbation entries. Query patterns shifted predictably: treatment simulations increased drug library queries; perturbation experiments elevated perturbation atlas queries. The LLM-driven agents dynamically retrieve relevant knowledge based on current context rather than relying on a fixed subset.

The systematic CD8$^+$ T cell overestimation and B cell underestimation observed across all models (\figref{fig:2}C; \figref{fig:5}D) also find their explanation here. The simulation's action space (proliferate, migrate, secrete, apoptose, rest) captures cytotoxic immune functions well but lacks explicit modules for humoral immunity: B cell differentiation, antibody secretion, and germinal center dynamics are absent. B cells consequently have limited functional roles and tend to be outcompeted by the more behaviorally active CD8$^+$ T cells, whose proliferative advantage is amplified by energy-based mechanics that reward successful immune engagement. This architectural bias, not any deficiency in LLM reasoning, accounts for the consistent compositional skew and identifies B cell functional modules as a priority for future development.

In summary, LLM-driven decision-making produces biologically plausible behaviors through cell cycle control that maintains quiescent populations, multi-pathway signal propagation that captures downstream perturbation effects, and context-dependent knowledge base utilization that adapts to experimental conditions. The anti-TGF$\beta$ failure case shows that simulation accuracy is bounded by knowledge base quality, pointing to a clear path for iterative improvement.
