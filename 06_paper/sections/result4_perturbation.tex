In baseline simulations and cross-cancer comparisons, Agent and Rules produced qualitatively similar outcomes. This raised a question: does LLM-driven decision-making offer any mechanistic advantage over explicit rule programming? We hypothesized that the differentiator would surface under genetic perturbations whose effects propagate through intermediate signaling cascades outside the rule decision boundary.

The rule-based engine evaluates CD8$^+$ T cell attack decisions using exactly four checkpoint pathway variables: net activation = TCR + CD28 $-$ PD-1 $-$ CTLA-4 $>$ 0.5. Genes outside these four variables are invisible to Rules by design. This constraint creates a controlled experiment: any response from Agent to a perturbation outside the rule boundary must arise from the LLM reasoning beyond the explicit decision logic.

We simulated seven gene knockouts (KOs) in two categories (\figref{fig:4}A; \tabref{tab:perturbation}). ``Direct'' KOs targeted checkpoint receptors whose pathway values enter the attack threshold: PD-1 KO and CTLA-4 KO. ``Indirect'' KOs targeted genes whose effects on tumor immunity are mediated through intermediate signaling outside the rule boundary: IFN-$\gamma$ KO, TGF-$\beta$ KO, TP53 KO, BRCA1 KO, and IL-2 KO. Checkpoint KOs drove the strongest tumor suppression (PD-1 KO: TR = 0.670 $\pm$ 0.09; CTLA-4 KO: TR = 0.630 $\pm$ 0.08); oncogene-related KOs (TP53, BRCA1) pushed tumor ratios above 1.0. Rules responded to direct KOs with even more aggressive clearance (PD-1 KO: TR = 0.469 $\pm$ 0.07; CTLA-4 KO: TR = 0.439 $\pm$ 0.08) but was nearly unchanged from baseline for all five indirect KOs ($\Delta$ = $+$0.3\%).

Each perturbation produced a distinct immune remodeling signature in Agent mode (\figref{fig:4}B). IFNG\_KO produced the most dramatic reshaping, with reduced CD8$^+$ T and NK cell proportions, while PD1\_KO and CTLA4\_KO enhanced CD8$^+$ T cell activity.

The IFN-$\gamma$ KO result is the most mechanistically informative (\figref{fig:4}C). Agent simulations showed tumor expansion (TR = 1.092 $\pm$ 0.05) from impaired immune surveillance; Rules applied a fixed suppression regardless of immune context (TR = 0.852 $\pm$ 0.03). The difference was highly significant (P = 0.005). Immune composition under IFNG\_KO confirmed reduced CD8$^+$ T and NK cell proportions consistent with IFN-$\gamma$ pathway disruption (\figref{fig:4}D). IFN-$\gamma$ does not appear in the rule-based attack threshold. Its effect on CD8$^+$ T cell function runs through the JAK-STAT1 signaling cascade; when that cascade is ablated, the rule engine sees no change in its input variables and behaves identically to baseline. The LLM-driven agent, by contrast, integrates the full pathway context---including the zeroed IFN-$\gamma$/JAK-STAT1 signal---and adjusts its decisions accordingly.

A scatter plot of Agent versus Rules sensitivity for each KO confirmed this pattern (\figref{fig:4}E). Direct KOs (PD-1, CTLA-4) elicited tumor reduction from both modes, with Rules showing larger effects. Indirect KOs (IFN-$\gamma$, TGF-$\beta$, TP53, BRCA1, IL-2) clustered along the y-axis at x $\approx$ 0; only Agent responded, while Rules remained inert.

Phenocopy analysis provided additional biological support (\figref{fig:4}F--H). PD1\_KO and anti-PD-1 treatment produced highly correlated cell-type proportion profiles ($r = 0.86$; \figref{fig:4}F), confirming that Agent captures the mechanistic link between checkpoint gene disruption and immunotherapy response. CTLA4\_KO versus anti-CTLA-4 showed moderate correlation ($r = 0.56$; \figref{fig:4}G); TGFB1\_KO versus anti-TGF$\beta$ showed weak correlation ($r = 0.31$; \figref{fig:4}H), consistent with TGF-$\beta$'s more complex, pleiotropic biology. A summary of phenocopy correlations (\figref{fig:4}I) confirmed that drug--KO concordance scales with pathway specificity: highly targeted checkpoint interventions produce strong phenocopy, while pleiotropic targets yield weaker correspondence.

Genetic perturbation sensing thus emerges as a capability unique to LLM-driven cell agents. Rule-based models excel at executing well-defined decision logic and indeed produce stronger responses to direct pathway perturbations. They are, by the same token, limited to the variables explicitly encoded in their rules. LLM-driven agents, drawing on biological knowledge embedded in their training data and the structured pathway information from the knowledge bases, can integrate the full pathway state and respond to perturbations that lie outside the rule-based decision boundary.
