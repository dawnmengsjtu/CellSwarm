\cellswarm{} shows that large language models can serve as the cognitive core of cell-level agents in tumor microenvironment simulations. The framework recapitulates TNBC composition, generalizes across six cancer types, and detects indirect genetic perturbations that rule-based models cannot sense. These results also expose specific limitations that bound the current claims and point toward concrete next steps.

\paragraph{Conservative decision bias.}
The most consistent behavioral signature of LLM-driven agents is conservatism. Under direct checkpoint knockouts, Rules produced 45--48\% tumor reduction while Agent achieved 29--33\%; anti-CTLA-4 yielded $\sim$10\% reduction in Rules versus 3--4\% in Agent mode. This gap likely reflects probabilistic hedging: when pathway signals are ambiguous, the LLM distributes probability mass across actions rather than committing to a single deterministic outcome. The bias is not uniformly detrimental. In the IFN-$\gamma$ knockout experiment, Agent correctly inferred impaired immune surveillance and allowed tumor expansion ($+$15.7\%), while Rules applied its fixed suppression regardless of immune context. Conservatism dampens direct perturbation responses but preserves sensitivity to indirect ones, a trade-off that parallels the redundancy and threshold-gating mechanisms of real immune signaling \citep{Chen2017}.

\paragraph{Simulation duration.}
All experiments used 30 time steps. This was sufficient to establish stable cell-type proportions and to separate Agent from Rules and Random modes, but it may be too short for slower biological processes. The non-significant difference between early and late anti-PD-1 treatment ($P = 0.30$; \figref{fig:3}H) could reflect insufficient time for population-level immune remodeling rather than a genuine absence of timing effects. The stable CD8$^+$/Treg ratios observed under perturbation may represent transient plateaus rather than true steady states. Sensitivity analyses at 60--100 steps are needed to resolve this ambiguity.

\paragraph{Validation scope.}
The validation strategy was uneven across cancer types. TNBC simulations were benchmarked against patient-level single-cell RNA-seq data from GSE176078 \citep{Wu2021}, providing quantitative ground truth. The five additional cancer types were evaluated only for directional consistency with literature-reported immune phenotypes. The CD8$^+$/Treg gradient from hot to cold tumors (\figref{fig:3}D) is encouraging, but it partly reflects initial conditions encoded in the Cancer Atlas knowledge base. The simulation's contribution lies in dynamically maintaining and elaborating these differences over 30 steps of cell--cell interaction, not in discovering them de novo. Patient-level transcriptomic validation for each cancer type remains an open task.

\paragraph{Systematic compositional bias.}
CD8$^+$ T cell overestimation ($+$44--87\%) and B cell underestimation ($-$54--80\%) persisted across all eight decision-making backends, including Rules (\figref{fig:5}D). The consistency across architecturally distinct models points to a structural cause rather than an LLM-specific reasoning deficit. The simulation's action space (proliferate, migrate, secrete, attack, apoptose, rest) captures cytotoxic functions well but lacks modules for humoral immunity. B cell differentiation into plasma cells, antibody secretion, and germinal center dynamics are absent. Without these functional roles, B cells are outcompeted by CD8$^+$ T cells whose proliferative advantage is amplified by energy-based mechanics. Adding B cell--specific modules is a priority for the next version.

\paragraph{Incomplete ablation.}
Knowledge base ablation was performed only in the baseline TNBC scenario (\figref{fig:5}E--F). Removing the Cancer Atlas was catastrophic (JS: 0.144 $\to$ 0.272); removing the Drug Library or Pathway KB had negligible baseline effects. These results establish the Cancer Atlas as the critical component for compositional accuracy but leave open whether the Drug Library and Pathway KB become essential under treatment and perturbation conditions, respectively. Condition-specific ablations (Drug Library $\times$ treatment, Pathway KB $\times$ perturbation) are planned.

\paragraph{Relationship to prior work.}
\cellswarm{} draws on the insight from Park et al.\ \citep{Park2023} that LLM-driven agents with memory and perception can produce emergent collective behaviors. The adaptation from social to biological simulation required replacing general world knowledge with structured, domain-specific knowledge bases, a substitution that proved essential given the Cancer Atlas ablation results. PhysiCell \citep{Ghaffarizadeh2018} and its ecosystem \citep{Macklin2012, Metzcar2019} provide physically grounded simulation with realistic cell mechanics, oxygen transport, and tissue-scale spatial dynamics that \cellswarm{} does not attempt. The two approaches are complementary: PhysiCell excels at biophysical fidelity with fixed decision rules; \cellswarm{} offers flexible cognition with simplified physics. A natural integration would embed LLM-driven decision modules within PhysiCell's mechanical framework. The approach of Sims et al.\ \citep{Sims2025} occupies an intermediate position, using LLMs at model construction time but not at runtime. \cellswarm{} moves the LLM into the simulation loop, giving each cell access to biological reasoning at every decision point. The perturbation results suggest that runtime reasoning captures phenomena that compile-time rule extraction cannot, such as indirect knockout sensing through pathway-level context integration.

\paragraph{Limitations.}
Several limitations constrain the current claims. LLM hallucination remains a risk; we mitigate it through structured prompts and a finite action space, but cannot eliminate it. Computational cost is non-trivial: each simulation consumes $\sim$2.8M tokens and runs for $\sim$20 minutes, limiting throughput for large-scale parameter sweeps. The spatial model is a 2D grid without mechanical forces, precluding simulation of tissue architecture, cell deformation, or interstitial fluid dynamics. The anti-TGF$\beta$ result, a 16-fold overestimate of clinical response (\figref{fig:6}C), traces to a Drug Library entry that conflates TGF-$\beta$'s pleiotropic effects into a simplified direct anti-tumor mechanism. This is a knowledge base quality issue, correctable without architectural changes, but it illustrates how simulation accuracy is bounded by the fidelity of the input knowledge.

\paragraph{Future directions.}
The most immediate application is patient-specific digital twins. The current framework already shows that swapping a single knowledge base entry changes the simulated cancer type; populating the Cancer Atlas with a patient's own single-cell transcriptomic profile and mutational landscape could generate personalized treatment predictions. The gap between this vision and clinical utility is substantial: it requires validated knowledge bases, calibration against longitudinal data, and rigorous uncertainty quantification. Beyond single-tumor modeling, multi-organ agent systems could capture metastatic dissemination and systemic immune responses. Extension to three-dimensional spatial grids and integration with PhysiCell's physics engine would address the current mechanical simplifications. These directions are feasible within the existing architecture; the knowledge base modularity that enables cross-cancer generalization also provides the entry point for richer biological content.
