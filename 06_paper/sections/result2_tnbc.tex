We simulated triple-negative breast cancer (TNBC) using CellSwarm and compared the output against single-cell RNA sequencing (scRNA-seq) data from a published cohort (GSE176078; Wu et al., 2021). Simulations began with 500 cells (tumor cells, CD8$^+$ T cells, macrophages, NK cells, regulatory T cells (Tregs), and B cells) seeded on a 500 $\times$ 500 two-dimensional grid with six diffusible signal fields (\figref{fig:2}A). Each run lasted 30 time steps under three decision-making modes: Agent (LLM-driven, DeepSeek), Rules (deterministic), and Random (uniform), with five independent seeds per condition.

The immune-to-tumor ratio (ITR) at the final step offered a first readout of whether each mode preserved a realistic balance between compartments (\figref{fig:2}B). Agent landed at 1.51 $\pm$ 0.11, nearly identical to the scRNA-seq reference of 1.49. Rules ran slightly higher (1.69 $\pm$ 0.06), reflecting more aggressive clearance. Random collapsed to 0.76 $\pm$ 0.06: without coordinated decisions, immune populations depleted and tumors dominated.

Cell-type proportions revealed finer differences (\figref{fig:2}C). Agent matched the real TNBC composition closely for tumor cells (40.0\% simulated vs.\ 40.1\% real), macrophages (15.1\% vs.\ 15.1\%), and Tregs (6.1\% vs.\ 6.3\%). CD8$^+$ T cells were overestimated (24.6\% vs.\ 17.1\%) and B cells underestimated (6.1\% vs.\ 17.2\%), a skew that persisted across every LLM backend we tested (\tabref{tab:model_benchmark}; Table~S1), pointing to an architectural limitation of the simulation rather than any single model's deficiency. Stacked composition profiles confirmed that Agent and Rules both approximated the real tissue distribution, while Random diverged substantially (\figref{fig:2}D). Per-cell-type absolute errors reinforced the pattern: Random showed the largest deviations, especially for tumor and CD8$^+$ T cells, whereas Agent and Rules errors tracked each other across most types (\figref{fig:2}E).

Ecological diversity provided a complementary lens. The Shannon diversity index ($H'$) for the real TNBC reference was 1.56 (theoretical maximum for six types: 1.79). Agent ($H' = 1.54 \pm 0.02$) and Rules ($H' = 1.57 \pm 0.01$) held steady near this value across all 30 steps (\figref{fig:2}F), meaning all six cell types coexisted in proportions consistent with the observed ecological balance. Random, by contrast, drifted downward to $H' = 1.26 \pm 0.05$ by step 30 as immune populations selectively disappeared.

We then benchmarked eight decision-making backends, six LLMs (DeepSeek, GLM-4-Flash, Qwen-Turbo, Qwen-Plus, Qwen-Max, Kimi-K2.5), hand-coded Rules, and Random, to identify the optimal backbone (\figref{fig:2}G; \tabref{tab:model_benchmark}). Two performance tiers emerged. Tier~1 (JS $< 0.16$) grouped DeepSeek (JS $= 0.144 \pm 0.001$), Rules ($0.146 \pm 0.001$), GLM-4-Flash ($0.147 \pm 0.002$), and Qwen-Turbo ($0.156 \pm 0.004$). Tier~2 contained Qwen-Plus ($0.210 \pm 0.007$), Qwen-Max ($0.212 \pm 0.018$), and Kimi-K2.5 ($0.219 \pm 0.002$). Random sat highest at $0.263 \pm 0.017$. Model scale did not predict performance: Qwen-Max, the largest model tested, underperformed the smaller Qwen-Turbo; instruction-following fidelity, not raw reasoning capacity, appears to govern simulation quality. Individual seed trajectories for all eight models are shown in Figure~S1. We selected DeepSeek as the default backbone for all subsequent experiments.

To place the TNBC results in a broader immunological context, we computed CD8$^+$/Treg ratios across six cancer types simulated with the same framework (\figref{fig:2}H). The gradient tracked expectations: immunologically ``hot'' tumors scored high (CRC-MSI-H: $4.65 \pm 0.06$; TNBC: $4.03 \pm 0.06$; Melanoma: $3.48 \pm 0.06$) and ``cold'' tumors scored low (NSCLC: 1.43; Ovarian: $1.10 \pm 0.02$; CRC-MSS: 1.00). Switching only the Cancer Atlas knowledge base entry, with no code or parameter changes, produced this separation, previewing the cross-cancer generalization explored in the next section.

CellSwarm thus recapitulates TNBC microenvironment composition with fidelity comparable to hand-coded rules (JS divergence 0.144 vs.\ 0.146) and far exceeding random decisions ($P = 0.012$, Mann--Whitney $U$ test, $n = 5$ seeds per condition). The framework maintains ecological diversity and a realistic immune-to-tumor balance without explicit programming of cell--cell interactions: the LLM's biological knowledge, structured through domain-specific knowledge bases, suffices to generate emergent, biologically plausible TME dynamics.
